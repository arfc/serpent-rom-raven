\documentclass{article}

\usepackage{xspace}

\usepackage[margin=0.5in]{geometry}

%% `Elsevier LaTeX' style
\bibliographystyle{elsarticle-num}
%%%%%%%%%%%%%%%%%%%%%%%

%%%% packages and definitions (optional)
\usepackage{placeins}
\usepackage{booktabs} % nice rules (thick lines) for tables
\usepackage{microtype} % improves typography for PDF
\usepackage{hhline}
\usepackage{amsmath}

%\usepackage[demo]{graphicx}
%\usepackage{caption}
%\usepackage{subcaption}

\usepackage{booktabs}
\usepackage{threeparttable, tablefootnote}

\usepackage{tabularx}
\newcolumntype{b}{>{\hsize=1.0\hsize}X}
\newcolumntype{s}{>{\hsize=.5\hsize}X}
\newcolumntype{m}{>{\hsize=.75\hsize}X}

\newcommand{\Cyclus}{\textsc{Cyclus}\xspace}%
\newcommand{\Cycamore}{\textsc{Cycamore}\xspace}%

% tikz %
\usepackage{tikz}
\usetikzlibrary{positioning, arrows, decorations, shapes}

\usetikzlibrary{shapes.geometric,arrows}
\tikzstyle{process} = [rectangle, rounded corners, minimum width=3cm, minimum height=1cm,text centered, draw=black, fill=blue!30]
\tikzstyle{object} = [ellipse, rounded corners, minimum width=3cm, minimum height=1cm,text centered, draw=black, fill=green!30]
\tikzstyle{arrow} = [thick,->,>=stealth]

% hyperref %
\usepackage[hidelinks]{hyperref}
% after hyperref %
\usepackage{cleveref}
\usepackage{datatool}
\usepackage[acronym,toc]{glossaries}
\include{acros}

\makeglossaries

\begin{document}
\title{CYCUS Reactor Module Development with Reduced-Order-Model Generation from RAVEN}

\date{}                     %% if you don't need date to appear



\section{Introduction}

The basic premise of this project is to create a quick, modular
reactor depletion model for various reactors in \Cyclus. This is done
by using two codes, RAVEN \cite{alfonsi_raven_2013} and SERPENT \cite{leppanen_serpent-a_2013} . SERPENT is a monte-carlo
reactor physics burnup calculation code, and RAVEN is a parametric
and probabilistic analysis tool. The goal of this project is to 
create an infrastructure to implement a RAVEN \gls{ROM} into a \Cyclus
reactor module to do depletion calculations of the fuel, and calculate
the keff of the core. 

In this development, the \gls{MSBR} \cite{robertson_conceptual_1971}
reactor \gls{ROM} is generated.

\section*{Thrust 1. Generate SERPENT output files in varying input space}
To generate a \gls{ROM}, large amounts of data tends to make the
\gls{ROM} more accurate. Before we start, the following is the input and 
output space for the SERPENT \gls{ROM}:

\begin{center}
\begin{tabular}{ c c }
    \hline
    Input Space & `Fresh' Composition , Depletion time \\
    \hline
    Output Space & `Depleted Composition', BOC $k_{eff}$, EOC $k_{eff}$ \\
    \hline
\end{tabular}
\end{center} 

Luckily, Andrei has been working on on-line reprocessing in \gls{MSBR}
and already running multiple SERPENT runs (!!!Make this less bro-y).
We obtained the SERPENT input and output data from Andrei and converted
into a csv file, with the given input and output space. (hdf5\_to\_csv.py)
Table \ref{tab:andrei} lists the datasets in the hdf5 generated by Andrei,
with details. Note that the depletion time in Andrei's SERPENT runs are
kept constant at three days. The hdf5 file is curated to a csv file to
the form shown in \ref{tab:csv}.
\FloatBarrier

\begin{table}
\begin{center}
\begin{tabular}{ c c }
    \hline
    Andrei's data & Details \\
    \hline
    core adensity after reproc & The input for SERPENT run (the composition that gets depleted) \\ 
    core adensity before reproc & The depleted composition of previous core adensity after reproc \\
    keff BOC & keff of `core adensity after reproc' \\
    keff EOC & keff of `core adensity before reproc' \\
    Th tank adensity & Composition in thorium tank \\
    iso codes & isotope codes array \\
    noble adensity & noble gases composition \\
    tank adensity & composition in Pa tank \\
    \hline
\end{tabular}
\end{center}
\label{tab:andrei}
\end{table}

\begin{table}

\begin{center}
\begin{tabular}{ c c }
    \hline
    Curated dataset & Method \\
    \hline
    `Fresh' composition & all but the last element in core adensity after reproc  \\
    Depletion time & array of 3's with length of keff BOC \\
    BOC keff & BOC keff \\
    Depleted composition & all but the first element in core adensity before reproc \\  
    EOC keff & EOC keff \\
    \hline
\end{tabular}
\label{tab:csv}
\end{center}
\end{table}


\subsection*{Aside: SERPENT-RAVEN interface}

By creating the SERPENT-RAVEN interface, RAVEN can run SERPENT
with varying parameters to generate SERPENT run data. For example,
a user can choose a variable (depletion time, U-233 Composition, etc.)
and have RAVEN choose its value by sampling from a distribution. 
In a RAVEN run, RAVEN will run multiple runs of SERPENT, save the 
data in csv, dump all the outputs of SERPENT in a separate directory.
An example of this is shown in: 


\section*{Thrust 2. Generate \gls{ROM} of SERPENT}


\section*{Thrust 3. Validate \gls{ROM} calculation with SERPENT runs}


\section*{Thrust 4. Implement \gls{ROM} to \Cyclus Module}


\pagebreak

\end{document}
