\documentclass{article}

\usepackage{xspace}

\usepackage[margin=0.5in]{geometry}

%% `Elsevier LaTeX' style
\bibliographystyle{elsarticle-num}
%%%%%%%%%%%%%%%%%%%%%%%

%%%% packages and definitions (optional)
\usepackage{placeins}
\usepackage{booktabs} % nice rules (thick lines) for tables
\usepackage{microtype} % improves typography for PDF
\usepackage{hhline}
\usepackage{amsmath}

%\usepackage[demo]{graphicx}
%\usepackage{caption}
%\usepackage{subcaption}

\usepackage{booktabs}
\usepackage{threeparttable, tablefootnote}

\usepackage{tabularx}
\newcolumntype{b}{>{\hsize=1.0\hsize}X}
\newcolumntype{s}{>{\hsize=.5\hsize}X}
\newcolumntype{m}{>{\hsize=.75\hsize}X}

\newcommand{\Cyclus}{\textsc{Cyclus}\xspace}%
\newcommand{\Cycamore}{\textsc{Cycamore}\xspace}%

% tikz %
\usepackage{tikz}
\usetikzlibrary{positioning, arrows, decorations, shapes}

\usetikzlibrary{shapes.geometric,arrows}
\tikzstyle{process} = [rectangle, rounded corners, minimum width=3cm, minimum height=1cm,text centered, draw=black, fill=blue!30]
\tikzstyle{object} = [ellipse, rounded corners, minimum width=3cm, minimum height=1cm,text centered, draw=black, fill=green!30]
\tikzstyle{arrow} = [thick,->,>=stealth]

% hyperref %
\usepackage[hidelinks]{hyperref}
% after hyperref %
\usepackage{cleveref}
\usepackage{datatool}
\usepackage[acronym,toc]{glossaries}
\include{acros}

\makeglossaries

\begin{document}
\title{CYCUS Reactor Module Development with Reduced-Order-Model Generation from RAVEN}

\date{}                     %% if you don't need date to appear



\section{Introduction}

The basic premise of this project is to create a quick, modular
reactor depletion model for various reactors in \Cyclus. This is done
by using two codes, RAVEN \cite{alfonsi_raven_2013} and SERPENT \cite{leppanen_serpent-a_2013} . SERPENT is a monte-carlo
reactor physics burnup calculation code, and RAVEN is a parametric
and probabilistic analysis tool. The goal of this project is to 
create an infrastructure to implement a RAVEN \gls{ROM} into a \Cyclus
reactor module to do depletion calculations of the fuel, and calculate
the keff of the core. 

In this development, the \gls{MSBR} \cite{robertson_conceptual_1971}
reactor \gls{ROM} is generated.

\section*{Thrust 1. Generate SERPENT output files in varying input space}
To generate a \gls{ROM}, large amounts of data tends to make the
\gls{ROM} more accurate. Before we start, the following is the input and 
output space for the SERPENT \gls{ROM}:

\begin{center}
\begin{tabular}{ l l }
    \hline
    Input Space & `Fresh' Composition , Depletion time \\
    \hline
    Output Space & `Depleted Composition', BOC $k_{eff}$, EOC $k_{eff}$ \\
    \hline
\end{tabular}
\label{tab:space}
\end{center} 

%%Tikz diagram of workflow

\tikzstyle{decision} = [diamond, draw, fill=blue!20, 
    text width=5em, text badly centered,  inner sep=0pt]
\tikzstyle{block} = [rectangle, draw, fill=blue!20, 
    text width=8em, text centered, rounded corners, minimum height=3em]
\tikzstyle{buff} = [rectangle, draw, fill=green!20, 
    text width=8em, text centered, rounded corners, minimum height=3em]
\tikzstyle{kernal} = [rectangle, draw, fill=red!20, 
    text width=, text centered, rounded corners, minimum height=3em]
\tikzstyle{file} = [rectangle, draw, fill=green!20, 
    text width=, text centered, rounded corners, minimum height=3em]
\tikzstyle{line} = [draw, -latex']
\tikzstyle{dummy} = [rectangle]
\tikzstyle{cloud} = [draw, text centered, ellipse,fill=blue!20, text width=,
    minimum height=2em]

\usetikzlibrary{calc}

\begin{figure}[!ht]
\begin{tikzpicture}[node distance = 2.5cm and 2cm, auto]
    % Place nodes
    \node [kernal] (RAVEN) {RAVEN};
    \node [cloud, below of=RAVEN] (int) {RAVEN-SERPENT Interface};
    \node [kernal, below of=int] (SERPENT) {SERPENT};
    \begin{scope}[node distance=6cm]
    \node [file, right of=SERPENT] (data) {SERPENT output data files (.csv)};
    \end{scope}
    \begin{scope}[node distance= 12cm]
    \node [file, right of=RAVEN] (ext) {External database (Andrei)};
    \node [cloud, right of=int] (cur) {Data curation to fit in RAVEN};
    \end{scope}
    
    \node [kernal, below of=data] (proc) {RAVEN (Process to one csv file) };
    \node [file, below of=proc] (database) {Compilation of SERPENT run data};
    \node [kernal, below of=database] (romgen) {RAVEN (ROM generation)};
    \node [file, below of=romgen] (rom) {Created SERPENT ROM};

    \path [line] (RAVEN) -- (int);
    \path [line] (int) -- (SERPENT);
    \path [line] (SERPENT) -- (data);
    \path [line] (data) -- (proc);
    \path [line] (ext) -- (cur);
    \path [line] (cur) |- (database);
    \path [line] (proc) -- (database);
    \path [line] (database) -- (romgen);
    \path [line] (romgen) -- (rom);
    
    \end{tikzpicture}
\caption{Workflow of Creating the \gls{ROM} for SERPENT. \textcolor{red}{Code},
         \textcolor{green}{Files}, \textcolor{blue}{Process}}
\end{figure}

\subsection*{Aside: Using External datasets}

Luckily, Andrei has been working on on-line reprocessing in \gls{MSBR}
and already running multiple SERPENT runs (!!!Make this less bro-y).
We obtained the SERPENT input and output data from Andrei and converted
into a csv file, with the given input and output space. (andrei\_saltproc\_run.py)
Table \ref{tab:andrei} lists the datasets in the hdf5 generated by Andrei,
with details. Note that the depletion time in Andrei's SERPENT runs are
kept constant at three days. The hdf5 file is curated to a csv file to
the form shown in \ref{tab:csv}.
\FloatBarrier

\begin{table}
\begin{center}
\begin{tabular}{ l l }
    \hline
    Andrei's data & Details \\
    \hline
    core adensity after reproc & The input for SERPENT run (the composition that gets depleted) \\ 
    core adensity before reproc & The depleted composition of previous core adensity after reproc \\
    keff BOC & keff of `core adensity after reproc' \\
    keff EOC & keff of `core adensity before reproc' \\
    Th tank adensity & Composition in thorium tank \\
    iso codes & isotope codes array \\
    noble adensity & noble gases composition \\
    tank adensity & composition in Pa tank \\
    \hline
\end{tabular}
\end{center}
\caption{Andrei has an HDF5 file that stored all his Saltproc runs.}
\label{tab:andrei}
\end{table}

\begin{table}
\begin{center}
\begin{tabular}{ l l }
    \hline
    Curated dataset & Method \\
    \hline
    `Fresh' composition & all but the last element in core adensity after reproc  \\
    Depletion time & array of 3's with length of keff BOC \\
    BOC keff & BOC keff \\
    Depleted composition & all but the first element in core adensity before reproc \\  
    EOC keff & EOC keff \\
    \hline
\end{tabular}
\label{tab:csv}
\end{center}
\end{table}


\subsection*{Aside: SERPENT-RAVEN interface}


By creating the SERPENT-RAVEN interface, RAVEN can run SERPENT
with varying parameters to generate SERPENT run data. For example,
a user can choose a variable (depletion time, U-233 Composition, etc.)
and have RAVEN choose its value by sampling from a distribution. 
In a RAVEN run, RAVEN will run multiple runs of SERPENT, save the 
data in csv, dump all the outputs of SERPENT in a separate directory.
An example of this is shown in \textbf{msbr\_time.xml}.

\subsubsection{Example of Running SERPENT with RAVEN for Varying Input Composition}
In the Robertson paper \cite{robertson_conceptual_1971}, the
input compositions are given a range. All other isotope
compositions remain, constant, while the U233
mole\% ranges from 0.2\% to 0.3\%. With the U233 change,
Li-7 composition changes to make up
for the difference ( 71.7 - 71.8\%) .
These values are converted  to mass fraction
for SERPENT input and better explained in a table in table \ref{tab:comp}.
%%% CHECK AGAIN
\begin{table}
\begin{center}
\begin{tabular}{ l l }
    \hline
    Isotopes & Mass Fraction \\
    \hline
    Li-6 & 3.24e-4 \\
    F-19 & 45.41 \\
    Be-9 & 2.5 \\ 
    Th-232 & 43.59 \\
    U-233 & 0.73 - 1.09 \\
    Li-7 & 7.63 - 8\\
    \hline
\end{tabular}
\label{tab:comp}
\end{center}
\end{table}

This distribution is run where the U-233 composition was
sampled in the range uniformly in 40 steps 
(\textbf{linspace(0.73, 1.09, 40)}). The Li-7 composition
is given by the remainder mass fraction 
(eg. 100 - 3.24e-4 -45.41 - 2.5 - 43.59 - 0.73).
Along with the composition variation,
the depletion time is varied from 0 to 30 days, also
in 40 steps to take into account
different initial reprocessing frequencies.
This creates an input grid of 40X40 = 1600 inputs to
train the ROM in.

The datasets currently available are listed in table \ref{tab:data}.

\begin{table}
\begin{center}
\begin{tabular}{ l l l}
    \hline
    Available Datasets & Details & Number of SERPENT Runs \\
    \hline
    andrei saltproc run & Mixed, reprocessed input composition & 290 \\
    1600 startup composition deptime & from Robertson \cite{robertson_conceptual_1971}, range of input & 1600 = $40_{Composition} * 40_{Deptime}$ \\
    \hline
\end{tabular}
\label{tab:data}
\end{center}
\end{table}

\section*{Thrust 2. Generate \gls{ROM} of SERPENT}
Data from both Andrei's saltproc run and RAVEN-run SERPENT
input is put together into one csv (\textbf{serpent\_run\_data.csv}), and is used to train the ROM. The ROM generation input file
is \textbf{generate\_rom\_from\_csv.xml}. 
Two separate \glspl{ROM} were generated for criticality and depletion calculation. Preliminary tests
show that having separate ROMs for each function yielded better results, especially for criticality
calculations because it only has one output variable. 

\section*{Thrust 3. Validate \gls{ROM} calculation with SERPENT runs}


\section*{Thrust 4. Implement \gls{ROM} to \Cyclus Module}


\pagebreak

\end{document}
